\section{Results and Discussion}\label{sect:results-and-discussion}
In this section we will examine the results accomplished by every algorithm
(\cref{subsect:brief-algo-discussion}) and close with a summary and discussion
(\cref{subsect:discussion}). The final metrics are presented in two tables:
\cref{tab:chosen-forecasting-algo,tab:chosen-boundary-algo}.

\subsection{Brief discussion of every algorithm}\label{subsect:brief-algo-discussion}
\bigskip
{\large\uline{\textbf{Inapplicable Algorithms}}}\\
An algorithm is considered inapplicable, if it produces results that are hardly
usable for anomaly detection.
\begin{description}[style=unboxed,leftmargin=0cm]
    \item[TBAT/S] TBAT/S suffered from frequent crashes due to numerical instabilities.
    In conjunction with its long runtime and subpar performance it was sorted out.
    \item[OCSVM] \gls{ocsvm}s output is predominantly noisy. In \cref{fig:ocsvm-output},
    two examples (out of many) are shown, in which \gls{ocsvm} was unable to extract
    meaningful insights from the data. Its output might be improved by adapting
    the method from e.g.\ \textcite{GomezVerdejo.2011}. However, this is beyond
    the scope of this paper.
    \item[LOF] produced only a low number of detections, most of which
    (89) were false positives that (visually) seem unreasonable (see \cref{fig:lof-output}).
    Interestingly, it proved largely resistant to high value spikes.
    For real-world use however, the number of false negatives is too high.
    \item[DAGMM] was able to produce 32 true positive detections,
    which however seem likely coincidental (see \cref{fig:dagmm-output}), making
    the algorithm inadequate.
    \item[RRCF] While in simple, consistent time series, \gls{rrcf} is able to
    adept to and detect violations of cyclicity, complex data appears to confuse
    the method and therefore makes it inadequate in large part. Although grid-search
    has been performed for hyperparameter optimization, it is possible that
    results could be improved in future research.
\end{description}
\bigskip
{\large\uline{\textbf{Forecasting Algorithms}}}\\
\begin{description}[style=unboxed,leftmargin=0cm]
    \item[DeepAnT] In simple cases, DeepAnT is able to learn cyclicity of the data,
    and thereby uncover cyclicity violation (see \cref{fig:deepant-cyclicity}).
    In more complex cases however, its output error-terms \(e\) more closely
    \textit{reconstruct} the time series\footnote{\label{foot:err-reconstructs}When
    error-terms seem to reconstruct the input time series, this implies that the
    method is unable to make valuable predictions beyond e.g.\ inferring the mean.}
    (see \cref{fig:deepant-resemble}). Neither increasing the history window
    (that is used to predict the upcoming observation) nor increasing the number
    of training iterations were able to alleviate the problem. An interesting
    exception from this are change points, where predictions slowly adapt to the
    new mean of observations (see \cref{fig:deepant-changepoint}).
    \item[NBEATS] Other than DeepAnT, NBEATS seem unable to infer cyclicity from
    the data (see \cref{fig:nbeats-cyclicity}). For more complex time series,
    NBEATS' error-terms also resemble\cref{foot:err-reconstructs} the original
    time series data (see \cref{fig:nbeats-resembles}). Interestingly, value
    spikes are followed by a short decrease in prediction accuracy
    (see \cref{fig:nbeats-spike-impact}). The output is therefore often more
    chaotic than that of DeepAnT. While it may seem counterintuitive that NBEATS
    achieves higher scores than DeepAnT. This is related to three conditions:
    \begin{enumerate*}[a.)]
        \item as noted, value spikes cause more chaotic outputs,
        \item many anomalies for the \gls{nab} are point anomalies with high value spikes, and
        \item \gls{ndt}~\cite[cf.][]{Hundman.2018} (used in phase b.) of
        forecasting algorithms, see \cref{def:forecasting-based-algo}) is sensitive
        to relative changes in mean and standard deviation.
    \end{enumerate*}
    
    We therefore observe more detections in total which are also more likely to
    be true positives, because the \gls{nab} contains many such anomalies.

    Unfortunately, this comes at the cost of increasing the number of false positives
    significantly.
    \item[Holt-Winters] Holt-Winters proved to be even more unstable than N-BEATS,
    resulting in large prediction spikes (see \cref{fig:hw-broken}). Predictions also
    often included some trend that was not to be found within the actual data (see
    \cref{fig:hw-trend-instability}). As Holt-Winters was adopted from AtsPy,
    it is possible that numerical instabilities resulted from an issue within the library.
    \item[Auto-ARIMA] Like DeepAnT, Auto-ARIMA is unable to detect long-term
    cyclicity (see \cref{fig:arima-cyclicity}), also resulting in high number of
    false positive on time series with many spikes (see \cref{fig:arima-fp}).
    This might be improved by adding a seasonal component to ARIMA~\cite[cf.][]{Box.2016}.
    Unfortunately, the seasonal ARIMA model proved computationally demanding for
    the task of real-time anomaly detection.

    On more simple time series, ARIMA is able to detect even subtle irregularities
    (see \cref{fig:arima-subtle}). However, this is also a downside, because
    (from observation) real world time series contain many such irregularities
    that are \textit{not} considered anomalous.
    
    On more chaotic examples the algorithms error-terms (again) more closely
    resemble\cref{foot:err-reconstructs} the input time series (see \cref{fig:arima-resembles}).
    
    As a side note to the interested reader, predictions were generated as in-sample
    1-step-ahead forecasts using \mbox{\texttt{predict\_in\_sample(*args, **kwargs)}~\cite{Smith.2017}}.
    Out-sample n-step-ahead forecasts produced worse results for the task.
    \item[Prophet] While Prophet scored highest from the forecasting algorithms,
    it also produced the highest number of false positives. Although a large number
    (57) of them can be attributed to only a single time series (see \cref{fig:prophet-fp}),
    Prophet still did not perform well. Similar to other forecasting algorithms
    for complex examples, Prophets error-terms reproduced\cref{foot:err-reconstructs}
    the original time series --- yet with a quirk: Prophet assumed seasonality within
    the data where seasonality was absent. This resulted in a curvy output
    (see \cref{fig:prophet-curvy}). Besides that, Prophet was able to understand
    cyclicity in \textit{simple} time series (see \cref{fig:prophet-cyclicity}).
\end{description}
\bigskip
{\large\uline{\textbf{Boundary Algorithms}}}\\
\begin{description}[style=unboxed,leftmargin=0cm]
    \item[Nonparametric Dynamic Thresholding~\cite{Hundman.2018}] Although not
    considered as a stand-alone method at the beginning, \gls{ndt} proved to be
    very effective at extracting anomalies from forecasting algorithms.
    
    As often observed in this section, the output of most forecasting algorithms
    closely resembles the shape of the ground truth time series.
    
    Therefore, \gls{ndt} was added to the set of methods.
    
    \gls{ndt} tries to detect values in a dataset that cause the greatest change
    in mean and standard deviation when they are removed. In theory, it should
    therefore be able to identify regions not only with a high/low value but also
    with unusually high/low mean. From visual evaluation, \gls{ndt} reacts most
    strongly to value spikes. In some time series, this leads to a high number
    of false positives (see \cref{fig:khundman-fp}). Unfortunately, the dataset
    yields not a single anomaly that is detected by \gls{ndt} due to its unusual
    mean.
    \item[LSTM-AD] Like most algorithms discussed so far, LSTM-AD is vulnerable
    to time series that produce high value spikes and then jump back to approximately
    zero (see \cref{fig:lstmad-fp}). It can be observed that LSTM-AD is unable to detect
    cyclicity violations (see \cref{fig:lstmad-cyclicity}). Besides that, all
    detections from LSTM-AD are closely related to some spiking change in
    value.
    \item[CBLOF] \gls{cblof}s results are surprisingly good. It was able to adapt
    to most regularly spiking time series (see \cref{fig:cblof-cyclicity}), albeit having trouble
    with \textit{ir}regularly spiking time series (see \cref{fig:cblof-fp}). Most of the
    time however, good results are achieved (see e.g.\ \cref{fig:cblof-good}).
    \item[kNN] was slightly more robust to spike-anomalies, producing
    less false positives but also less true positives (see \cref{fig:spike-missed}).
    Although \gls{knn} has no notion of time, it was able to detect several
    (simple) cyclicity violations (see \cref{fig:knn-cyclicity}). Presumably,
    because of their unusual value ranges. Besides that, \gls{knn} was also able
    to detect some of the more challenging anomalies (see \cref{fig:knn-harder}).
    \item[Threshold Detector] The threshold detector was originally a \textbf{cheat}
    within the \gls{htm}-code\footnote{\raggedright\url{https://github.com/numenta/NAB/blob/757b586b35a939645dd92f6f7c1b126ebb3614f7/nab/detectors/numenta/numenta_detector.py}}
    by Numenta~\cite{Lavin.2015,Ahmad.2017}. It exploits the many point anomalies
    (spiking values) within the \gls{nab} to achieve higher scores.
    
    The detector works by remembering both the highest/lowest observed values. If a
    new point with higher/lower value is observed (and it is larger/smaller by some
    margin), an anomaly is reported.

    This simple method was able to achieve (by far) the highest F1-Score (see
    \cref{tab:results}) due to its very low false positive rate. Representative
    examples are given in \cref{fig:threshold-output}.
    \item[LSTM-ED] From visual examination, results look very similar to those
    achieved from \gls{cblof}. LSTM-ED suffers similarly from irregularly spiking
    datasets (see \cref{fig:lstmed-fp}). LSTM-ED is able to improve upon \gls{cblof}
    by detecting some of the more intricate anomalies (see \cref{fig:lstmed-harder}).
    The higher number of false positives from LSTM-ED can be linked to the set of
    artificial time series data in \gls{nab} that does not contain any anomalies.
    There, LSTM-ED produced 21 anomalies more than \gls{cblof} (see \cref{fig:lstmed-artificial}).
    \item[Autoencoder] The output of the \gls{ae} was often identical to that of LSTM-ED\@.
    The two algorithms differed only on about three of the 58 time series
    (see \cref{fig:lstm-vs-ae}).
    \item[Skyline] Skyline was unable to detect cyclicity dropouts (see \cref{fig:skyline-cyclicity}).
    It is robust to frequently spiking domains, as long as the spikes are so common,
    that they are within a \(\mu_t + 3 \times \sigma_t\), where \(\mu_t, \sigma_t\)
    are the moving mean and the moving standard deviation respectively.
    If observations deviate more (however regular that may be), a detection is recorded.
    This produced a significant number of false positives in some domains (see \cref{fig:skyline-fp}).
    \item[Numenta HTM] was able to score highest out of all algorithms,
    and slightly higher than Skyline because it produced less false positives. In simple
    time series, \gls{htm} was able to detect very simple cyclicity violations
    (see \cref{fig:htm-cyclicity,fig:htm-cyclicity2}). Also, \gls{htm} was more
    sensitive to small deviations than most competitors, enabling it to detect
    more subtle (see \cref{fig:htm-subtle}) anomalies, but also producing a larger number of false positives.
    The algorithm was not robust to datasets where spikes were strong but regular
    (see \cref{fig:htm-spikes}).
\end{description}

% Chosen Library Anomaly Detection & Forecasting
\begin{table}[h]\centering
    \ra{1.3}
    \resizebox{\textwidth}{!}{%
        \begin{tabular}{lll|cccc}
            Boundary Algorithms                                             & Standard Profile  & Punish high FP    & F1      & TP    & FP    & FN    \\\hline
            Numenta \gls{htm}~\cite{Ahmad.2017}                             & 59.3              & 21.9              & 0.47    & 82    & 153   & 34    \\
            Skyline                                                         & 57.8              & -6.4              & 0.37    & 87    & 273   & 29    \\
            \gls{ae}~\cite[499\psqq]{Goodfellow.2016}                       & 53.5              & 22.9              & 0.47    & 76    & 129   & 40    \\
            LSTM-ED~\cite{Malhotra.2016}                                    & 51.2              & 18.3              & 0.45    & 74    & 141   & 42    \\
            Numenta Threshold Detector~\cite{Ahmad.2017}                    & 50.1              & 44.9              & 0.65    & 65    & 19    & 51    \\
            \gls{knn}~\cite[16\psqq]{Murphy.2012}                           & 47.9              & 17.6              & 0.44    & 68    & 123   & 48    \\
            \gls{cblof}~\cite{He.2003}                                      & 47.7              & 19.6              & 0.46    & 68    & 114   & 48    \\
            \gls{ndt}~\cite{Hundman.2018}                                   & 46.2              & 9.69              & 0.42    & 68    & 140   & 48    \\
            LSTM-AD~\cite{Malhotra.2015}                                    & 36.3              & 17                & 0.42    & 51    & 78    & 65    \\
            Robust Random Cut Forest~\cite{Bartos.2019}                     & 34.3              & -14               & 0.29    & 54    & 207   & 62    \\
            \gls{dagmm}\cite{Zong.2018}                                     & 17.9              & -17.1             & 0.22    & 32    & 147   & 84    \\
            \gls{lof}~\cite{Breunig.2000}                                   & 14.2              & -7.9              & 0.21    & 24    & 89    & 91    \\
            \acrshort{ocsvm}~\cite{Schölkopf.1999,Tax.2004}                 & 1.2               & -0.5              & 0.02    & 1     & 5     & 115   \\
            \\
            Forecasting Algorithm                                           &                   &                   &         &       &       &       \\\hline
            Prophet~\cite{Taylor.2017}                                      & 48.0              & -26.5             & 0.3     & 78    & 319   & 38    \\
            Auto-ARIMA~\cite{Smith.2017}                                    & 47.6              & -19.7             & 0.32    & 77    & 287   & 39    \\
            Holt-Winters (additive model)~\cite{Winters.1960}               & 46.2              & -18.2             & 0.31    & 74    & 281   & 42    \\
            N-BEATS~\cite{Oreshkin.2020}                                    & 44.7              & -13.6             & 0.31    & 71    & 272   & 45    \\
            DeepAnT~\cite{Munir.2019}                                       & 36.9              & -2.5              & 0.34    & 57    & 166   & 59    \\
            TBAT/S                                                          & /                 & /                 & /     & /     & /     & /     \\
        \end{tabular}
    }
    \caption[NAB-Scores achieved by the algorithms]{
    (left) overview of scores achieved by two \gls{nab}-metric profiles.
    (right) F1-Score~\cite[183]{Murphy.2012}, true positives, false positives
    and false negatives respectively.}\label{tab:results}
\end{table}


\subsection{Discussion}\label{subsect:discussion}
