\section{Results and Discussion}\label{sect:results-and-discussion}
In this section we will briefly examine the results accomplished by every
of the 22 algorithms (\cref{tab:chosen-forecasting-algo,tab:chosen-boundary-algo})
selected for evaluation.\todo{22 algorithms?}

\bigskip
{\large\uline{\textbf{Inapplicable Algorithms}}}\\
An algorithm is considered inapplicable, if it is unable to produce results that
are hardly usable for anomaly detection.
\begin{description}[style=unboxed,leftmargin=0cm]
    \item[TBAT/S] TBAT/S suffered from frequent crashes due to numerical instabilities.
    In conjunction with its long runtime and subpar performance it was sorted out.
    \item[OCSVM] \gls{ocsvm}s output is predominantly noisy. In \cref{fig:ocsvm-output},
    two examples (out of many) are shown in which \gls{ocsvm} was unable to extract
    meaningful insights from the data. Its output might be improved by adapting
    the method from e.g.\ \textcite{GomezVerdejo.2011}. However, this is beyond
    the scope of this paper.
    \item[LOF] \gls{lof} produced only a low number of detections, most of which
    (89) were false positives that (visually) seem unreasonable (see \cref{fig:lof-output}).
    Interestingly, it proved largely resistant to high value spikes.
    For real-world use however, the number of false negatives is too high.
    \item[DAGMM] Although \gls{dagmm} was able to produce 32 true positive detections,
    most of them appear coincidental (see \cref{fig:dagmm-output}) making the 
    algorithm inadequate.
    \item[RRCF] While in simple, consistent time series, \gls{rrcf} is able to
    adept to and detect violations of cyclicity, complex data appears to confuse
    the method and therefore makes it inadequate in large part. Although grid-search
    has been performed for hyperparameter optimization, it is possible that
    results could have been improved.
\end{description}
\bigskip
{\large\uline{\textbf{Forecasting Algorithms}}}\\
\begin{description}[style=unboxed,leftmargin=0cm]
    \item[DeepAnT] In simple cases, DeepAnT is able to learn cyclicity of the data,
    and thereby uncover cyclicity violation (see \cref{fig:deepant-cyclicity}).
    In more complex cases however, the result more closely \textit{reconstructs}
    it (see \cref{fig:deepant-resemble}). Neither increasing the history window
    (that is used to predict the upcoming observation) nor increasing the number
    of training iterations were able to alleviate the problem. An interesting
    exception from this are change points, where predictions slowly adapt to the
    new mean of observations (see \cref{fig:deepant-changepoint}).
    \item[NBEATS] Other than DeepAnT, NBEATS seem unable to infer cyclicity from
    the data (see \cref{fig:nbeats-cyclicity}). For more complex time series,
    NBEATS also resembles the original time series data (see \cref{fig:nbeats-resembles}).
    Interestingly, value spikes are followed by a short decrease in prediction
    accuracy (see \cref{fig:nbeats-spike-impact}). The output is therefore often
    more chaotic than that of DeepAnT. While it may seem counterintuitive, that
    NBEATS achieves higher scores than DeepAnT. This is related to three conditions:
    \begin{enumerate*}[a.)]
        \item as noted, value spikes cause more chaotic outputs,
        \item many anomalies for the \gls{nab} are point anomalies with high value spikes, and
        \item Nonparametric Dynamic Thresholding~\cite[cf.][]{Hundman.2018}
        (see \cref{def:forecasting-based-algo}) is sensitive to relative changes
        in mean and standard deviation.
    \end{enumerate*}
    
    We therefore observe more detections in total which are also more likely to
    be true positives, because the \gls{nab} contains many such anomalies.

    Unfortunately, this comes at the cost of increasing the number of false positives
    significantly.
    \item[Holt-Winters] Holt-Winters proved to be even more unstable than N-BEATS,
    resulting in large prediction spikes (see \cref{fig:hw-broken}). Predictions also
    often included some trend that was not to be found within the actual data (see
    \cref{fig:hw-trend-instability}). As Holt-Winters was adopted from AtsPy,
    it is possible that numerical instabilities resulted from an issue within the library.
    \item[Auto-ARIMA] Unlike DeepAnT, Auto-ARIMA is unable to detect long-term
    cyclicity (see \cref{fig:arima-cyclicity}), also resulting in high number of
    false positive on time series with many spikes (see \cref{fig:arima-fp}).
    This might be improved by adding a seasonal component to ARIMA~\cite[cf.][]{Box.2016}.
    Unfortunately, the seasonal ARIMA model proved computationally demanding for
    the task of real-time anomaly detection.

    On more simple time series, ARIMA is able to detect even subtle irregularities
    (see \cref{fig:arima-subtle}). However, this is also a downside, because
    (from observation) real world time series contain many such irregularities
    that are \textit{not} considered anomalous.
    
    On more chaotic examples (again) the algorithms output more closely resembles
    the input time series (see \cref{fig:arima-resembles}).
    
    As a side note to the interested reader, predictions were generated as in-sample
    1-step-ahead forecasts using \mbox{\texttt{predict\_in\_sample(*args, **kwargs)}~\cite{Smith.2017}}.
    Out-sample n-step-ahead forecasts produced worse results for the task.
    \item[Prophet] While Prophet scored highest from the forecasting algorithms,
    it also produced the highest number of false positives. While a large number
    (57) of them can be attributed to only a single time series (see \cref{fig:prophet-fp}),
    Prophet did not perform well. Similar to other forecasting algorithms for complex examples, Prophet
    reproduced the original time series --- although with a quirk: Prophet
    assumed seasonality of the data, resulting in a curvy output (see \cref{fig:prophet-curvy}).
    Besides that, Prophet also was able to understand cyclicity in simple time series
    (see \cref{fig:prophet-cyclicity}), while results from more complex examples merely mirrored
    the original time series.
\end{description}\todo{Predictions from forecasting methods are actually prediction-errors}
\bigskip
{\large\uline{\textbf{Boundary Algorithms}}}\\
\begin{description}[style=unboxed,leftmargin=0cm]
    \item[Nonparametric Dynamic Thresholding~\cite{Hundman.2018}] Although not
    considered as a stand-alone method at the beginning, \gls{ndt} proved to be
    very effective at extracting anomalies from forecasting algorithms.
    
    As often observed in this section, the output of most forecasting algorithms
    closely resembles the shape of the ground truth time series.
    
    Therefore, \gls{ndt} was added to the set of methods.
    
    \gls{ndt} tries to detect values in a dataset that cause the greatest change
    in mean and standard deviation when they are removed. In theory, it should
    therefore able to identify regions not only with a high/low value but also
    with unusually high/low mean. From visual evaluation reacts most strongly to
    value spikes. In some time series, this leads to a high number of false positives
    (see \cref{fig:khundman-fp}). Unfortunately, the dataset yields not a single
    anomaly that is detected by \gls{ndt} due to its unusual mean.
    \item[LSTM-AD] Like most algorithms discussed so far, LSTM-AD is vulnerable
    to time series that produce high value spikes and then jump back to approximately
    zero (see \cref{fig:lstmad-fp}). It can be observed that LSTM-AD is unable to detect
    cycilicity violations (see \cref{fig:lstmad-cyclicity}). Besides that, all
    detections from LSTM-AD are closely related to some spiking change in
    value.
    \item[CBLOF] \gls{cblof}s results are surprisingly good. It was able to adapt
    to most regularly spiking time series (see \cref{fig:cblof-cyclicity}), albeit having trouble
    with \textit{ir}regularly spiking time series (see \cref{fig:cblof-fp}). Most of the
    time however, good results are achieved (see e.g.\ \cref{fig:cblof-good}).
    \item[LSTM-ED] From visual examination, results from LSTM-ED look very similar
    to those achieved from \gls{cblof}. LSTM-ED suffers similarly from irregularly
    spiking datasets (see \cref{fig:lstmed-fp}). LSTM-ED is able to improve upon \gls{cblof}
    by detecting some of the more intricate anomalies (see \cref{fig:lstmed-harder}). The
    higher number of false positives from LSTM-ED can be linked to the set of
    artificial time series data in \gls{nab} that does not contain any anomalies.
    There, LSTM-ED produced 21 anomalies more than \gls{cblof} (see \cref{fig:lstmed-artificial}).
    \item[kNN] Like most algorithms, \gls{knn} was able to detect the majority
    of spike-anomalies. Although \gls{knn} has no notion of time, it was able
    to detect several (simple) cyclicity violations (see \cref{fig:knn-cyclicity}). Presumably,
    because of their unusual value ranges. Besides that, \gls{knn} was also able
    to detect some of the more challenging anomalies (see \cref{fig:knn-harder}).
    \item[Threshold Detector] Interestingly, the threshold detector achieves high scores.
    This shows, that the majority of anomalies in the dataset are point anomalies 
    that spike higher than all previous observations. It was able to detect
    71 out of 112 anomalies.
    \item[Skyline] Skyline was unable to detect cyclicity dropouts. It is robust
    to frequently spiking domains, as long as the spikes are so common, that
    they are within a \(\mu_t + 3 \times \sigma_t^2\), where \(\mu_t, \sigma_t^2\) 
    are the moving mean and the moving standard deviation respectively. If observations
    deviate more (however regular that may be), a detection is recorded.
    This produces a significant number of false positives in some domains.
    \item[Numenta HTM] \gls{htm} was able to score highest out of all algorithms,
    and slightly higher than Skyline while producing less false positives. It
    was unable to detect (even visually very obvious) anomalies in highly
    fluctuating time series, where the height (and thereby mean) of spikes
    changes. It produced several false positives on datasets where spikes were
    strong but regular. Furthermore, it is hard to determine a rule, which datasets
    work and which produce a high number of false positives.

    \item[AutoEncoder] While the autoencoder was able to achieve a good score,
    evaluating its output visually proved challenging. In frequently spiking
    domains, it was robust to most spikes. Sometimes too robust. In a time series with a
    steady square wave however, it seemed to randomly detect observation with a
    high value spike. Cyclicity dropouts (\cref{app-fig:art_daily_flatmiddle,app-fig:art_daily_jumpsdown})
    were not detected.
    
    An in-depth discussion with carefully picked time series would be required for
    a better understanding its real-world behavior more. This is beyond the scope
    of this paper.
\end{description}



% Chosen Library Anomaly Detection & Forecasting
\begin{table}[h]\centering
    \ra{1.3}
    \resizebox{\textwidth}{!}{%
        \begin{tabular}{lll|cccc}
            Boundary Algorithms                                             & Standard Profile  & Punish high FP    & F1      & TP    & FP    & FN    \\\hline
            Numenta \gls{htm}~\cite{Ahmad.2017}                             & 59.3              & 21.9              & 0.41    & 82    & 153   & 34    \\
            Skyline                                                         & 57.8              & -6.4              & 0.33    & 87    & 273   & 29    \\
            Autoencoder~\cite[499\psqq]{Goodfellow.2016}                    & 51.9              & -5                & 0.32    & 80    & 260   & 36    \\
            LSTM-ED~\cite{Malhotra.2016}                                    & 51.2              & 18.3              & 0.45    & 74    & 141   & 42    \\
            Numenta Threshold Detector~\cite{Ahmad.2017}                    & 50.1              & 44.9              & 0.61    & 65    & 19    & 51    \\
            \gls{knn}~\cite[16\psqq]{Murphy.2012}                           & 47.9              & 17.6              & 0.42    & 68    & 123   & 48    \\
            \gls{cblof}~\cite{He.2003}                                      & 47.7              & 19.6              & 0.43    & 68    & 114   & 48    \\
            \gls{ndt}~\cite{Hundman.2018}                                   & 46.2              & 9.69              & 0.42    & 68    & 140   & 48    \\
            LSTM-AD~\cite{Malhotra.2015}                                    & 36.3              & 17                & 0.44    & 51    & 78    & 65    \\
            Robust Random Cut Forest~\cite{Bartos.2019}                     & 34.3              & -14               & 0.29    & 54    & 207   & 62    \\
            \gls{dagmm}\cite{Zong.2018}                                     & 17.9              & -17.1             & 0.26    & 32    & 147   & 84    \\
            \gls{lof}~\cite{Breunig.2000}                                   & 14.2              & -7.9              & 0.3     & 24    & 89    & 91    \\
            \acrshort{ocsvm}~\cite{Schölkopf.1999,Tax.2004}                 & 1.2               & -0.5              & 0.25    & 1     & 5     & 115   \\
            \\
            Forecasting Algorithm                                           &                   &                   &         &       &       &       \\\hline
            Prophet~\cite{Taylor.2017}                                      & 48.0              & -26.5             & 0.28    & 78    & 319   & 38    \\
            Auto-ARIMA~\cite{Smith.2017}                                    & 47.6              & -19.7             & 0.3     & 77    & 287   & 39    \\
            Holt-Winters (additive model)~\cite{Winters.1960}               & 46.2              & -18.2             & 0.29    & 74    & 281   & 42    \\
            N-BEATS~\cite{Oreshkin.2020}                                    & 44.7              & -13.6             & 0.29    & 71    & 272   & 45    \\
            DeepAnT~\cite{Munir.2019}                                       & 36.9              & -2.5              & 0.34    & 57    & 166   & 59    \\
            TBAT/S                                                          & /                 & /                 & /     & /     & /     & /     \\
        \end{tabular}
    }
    \caption[NAB-Scores achieved by the algorithms]{
    Left) overview of scores achieved by two \gls{nab}-metric profiles.
    \\Right) F1-Score~\cite[183]{Murphy.2012}, true positives, false positives
    and false negatives respectively.}\label{tab:results}
\end{table}